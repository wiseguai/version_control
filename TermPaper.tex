%%% Document Class

\documentclass[11pt]{article}


%%% Font Packages Providing Math Support

%\usepackage{FE-FourierX}
%\usepackage{FE-MinionPro}
%\usepackage{FE-AGaramondPro}
%\usepackage{FE-HypatiaSansPro}
%\usepackage{FE-Lmodern}
%\usepackage{FE-MathPazo}
%\usepackage{FE-EuPaHeCo}
%\usepackage{FE-MathPTMX}
%\usepackage{FE-CM-Bright}
%\usepackage{FE-CC-Euler}
%\usepackage{FE-UtopiaMathDesign}
%\usepackage[math,condensed]{FE-AntTor}
%\usepackage[math,light]{FE-AntTor}
%\usepackage[math,light,condensed]{FE-AntTor}
%\usepackage{FE-Euler}
%\usepackage{FE-EulerVM}
%\usepackage[small,text-hat-accent,euler-digits,icomma,OT1,LY1]{FE-EulerVM}
%\usepackage{FE-SFMath}
%\usepackage{FE-Arev}
%\usepackage{FE-Baskervald-X}
%\usepackage{FE-Charter-BT}
%\usepackage{FE-Computer-Concrete}
%\usepackage{FE-Erewhon}
%\usepackage{FE-GFSArtemisa}
%\usepackage{FE-GFSArtemisa-Euler}
%\usepackage{FE-GFSArtemisa-NeoHellenic}
%\usepackage{FE-Heuristica}
%\usepackage[math]{FE-Iwona}
%\usepackage[math,condensed]{FE-Iwona}
%\usepackage[math,light]{FE-Iwona}
%\usepackage[math,light,condensed]{FE-Iwona}
%\usepackage{FE-Iwona}
%\usepackage{FE-Kerkis}
%\usepackage{FE-KP-Sans-Serif}
%\usepackage[math,condensed]{FE-Kurier}
%\usepackage[math,light]{FE-Kurier}
%\usepackage[math,light,condensed]{FE-Kurier}
%\usepackage{FE-LX}
%\usepackage{FE-NewPX}
%\usepackage{FE-NewTX}
%\usepackage{FE-PX}
%\usepackage{FE-Stix}
%\usepackage{FE-TX}
%\usepackage{FE-URW-Nimbus-Roman}
%\usepackage{FE-URW-Schoolbook-L}
%\usepackage{FE-Utopia-Regular-Math-Design}


%%% Packages

\usepackage{amsmath}
\usepackage{bm}
\usepackage[pdftex]{graphicx}
\usepackage{pdfsync}
\usepackage{FE-Typing-Aids}
\usepackage{FE-Color}
\usepackage[draft,author={Frank Ernst}]{pdfcomment}
%\usepackage[final,author={Frank Ernst}]{pdfcomment}
\usepackage{todonotes}
\usepackage{multicol}
\usepackage{subcaption}

\newcommand{\dmac}{Department of Macromolecular Science and Engineering}
\title{Approaches to Improve Thermal Conductivity in Polymers: A
  Review}
\author{Ian Wise \\
  \small\dmac \\
\small{Case Western Reserve University}}

\date{18 October 2018}

\bibliographystyle{elsarticle-num}

\begin{document}

\maketitle

\textbf{Abstract}: Thermally conductive polymers and polymer composites
open new opportunities in reducing engineering constraints in thermal
control and electronic systems, especially where insulating properties
are preferred. The addition of a non-polymeric filler shows a
significant improvement in the overall thermal conductivity of polymers
over the unfilled compound. Current research into high thermal
conducting polymer composites seeks to develop a polymer-based, filled
compound with a highly efficient thermal conductivity without removing
the desirable properties of the original polymer. In this article, a
variety of materials used as fillers in polymer composites are reviewed,
as well as experimental results and observations of such fillers as part
of a polymer-based composite material.

\begin{multicols}{2}

\section{Introduction}

Thermal conductivity describes a material's ability to transfer heat.
Grundler et. al. note the general equation for thermal conductivity:

\begin{equation}
  \lambda = \emph{a} \cdot \rho \cdot C\textsubscript{p} 
\end{equation}

where a denotes the thermal diffusivity, \( \rho \) denotes the density, and
C\textsubscript{p}­ denotes the specific heat capacity.\cite{Grundler-2016}
However, as noted by Kumlutas et al., when working with polymer composites, it
becomes more efficient to utilize the geometric mean model for effective
thermal conductivity k:

\begin{equation}
k_c=k_f^\varphi k_m^{(1-\varphi)}
\end{equation}

where c denotes the composite, f denotes the filler, m denotes the
polymer matrix, and \( \varphi \) denotes the volumetric fraction of filler.\cite{Kumlutas-2003-113}
This model or one of its many derivatives (e.g. Agari's, Russell's, and
Maxwell's equations) is often used to contrast experimental data when
measuring the thermal conductivity of a polymer composite.

The sheer complexity of many polymers significantly reduces their
thermal conductivity. Varying degrees of crystallinity, a high molecular
weight, a large number of entropic defects, and other factors contribute
to a high level of phonon scattering within most commercialized polymers
and prevents the material from efficiently conducting heat. Research
into polymer composites containing a hybrid filler show the possibility
of increasing thermal conductivity from under 1 W/mK to over 100 W/mK;
however, this is highly dependent upon the material used as filler. In
this article, metallic, ceramic, and carbon-based fillers for polymer
composites will be discussed and compared.

\end{multicols}

\begin{figure}[t]
\begin{subfigure}{.5\textwidth}
  \centering
  \includegraphics[width=.8\linewidth]{./FIG/Boudenne.pdf}  
  \caption{Boudenne. \cite{Boudenne-2005-1545}}
  \label{fig:Boudenne}
\end{subfigure}
\begin{subfigure}{.5\textwidth}
  \centering
  \includegraphics[width=.8\linewidth]{./FIG/Zhou.pdf}  
  \caption{Zhou. \cite{Zhou-2007-1863}}
  \label{fig:Zhou}
\end{subfigure}
\caption{Two figures from the literature.}
\label{fig:fig}
\end{figure}

\begin{multicols}{2}
  
\section{Metallic Polymer Hybrids}

The use of metallic particles in polymer hybrids appears promising in

terms of raising thermal conductivity. Most metals themselves exhibit
thermal conductivities of over 100 W/mK. The addition of metals to
polymers, which have thermal conductivities of mostly less than 1 W/mK,
would thus improve the effectiveness of the material in that regard.
Kumlutas et al. examined powdered high-density polyethylene uniformly
mixed with aluminum powder. Previous theoretical models failed to
predict how the experimental results grew exponentially once a
volumetric Al wt\% of 10 was reached, with a maximum reading at over 3.5
W/mK at an Al wt\% of over 30.\cite{Boudenne-2005-1545}  Boudenne et al. combined
polypropylene with two sizes of copper particles. Results showed an
increase in thermal conductivity at medium filler volume percentage,
with a maximum thermal conductivity of about 2.25 W/mK read at about
40\% Cu.\cite{Boudenne-2005-1545}

\subsection{Effect of particle size on thermal conductivity}

The thermal conductivity of metallic polymer hybrids is significantly
determined by the structure of the filler being utilized. Boudenne, et.
al. experimented with two sizes of Cu particles, the first
(Cu\textsubscript{a}) measuring on average at about 25 µm in length, and
the second (Cu\textsubscript{b}) measuring on average at about 250 µm in
length. Results showed a significant increase in thermal conductivity in
the sample containing Cu\textsubscript{a} particles.\cite{Boudenne-2005-1545}  This is
expected as in general, higher thermal conductivities can be obtained
using smaller filler particles in metallic polymer hybrids due to a more
even dispersion of conducting particles in the polymer matrix.

\section{Ceramic Polymer Hybrids}

Ceramic-based polymer hybrids offer the opportunity for the use of
polymers in electronic applications in place of ceramic components such
as resistors, and where efficient heat dissipation is required. Zhou et
al. utilized high-density polyethylene (HDPE) reinforced with boron
nitride. Among common ceramics, boron nitride exhibits both a high
thermal conductivity of about 400 W/mK in the X-Y plane and temperature
resistance. Results show a general increase in thermal conductivity with
increasing filler volume fraction. At the highest tested volume fraction
of 35\%, the thermal conductivity was found to be 0.87 and 1.02 W/mK for
the melting and powder mixing methods, respectively.\cite{Zhou-2007-1863}  Lee, et.
al. mixed HDPE with aluminum nitride, boron nitride, wollastonite, and
silicon carbide fillers in both single filler and hybrid filler
configurations. Furthermore, several samples of the AlN and hybrid
filler composites were treated with a titanate coupling agent. Results
found a significant increase in thermal conductivity at filler contents
of between 50 and 75 vol\%. Better results than those of the above study
were found with the BN sample, which measured a thermal conductivity of
3.66 W/mK at 50 vol\%. For the single-filler AlN composites, the
surface-treated sample was found to have the highest measured thermal
conductivity per filler concentration, measuring at 2.14 W/mK at 60
vol\%. Of the three hybrid-filler composites tested, the AlN/SiC sample
measured the highest thermal conductivity. Surface treatment of the
hybrid-filler samples increased measured thermal conductivity by about
0.14 W/mK.\cite{Lee-2006-727}

\subsection{Effect of hybrid fillers on thermal conductivity}

Hybrid-filler polymer composites possess the unique challenge of finding
the most thermally efficient combination of single materials within the
filler. Lee, et. al. utilized the concept of maximum packing fraction
within their hybrid fillers in order to synthesize the most conductive
filler possible. At a specific volume fraction of wollastonite and SiC
in AlN, a hybrid formed which when mixed with a polymer best facilitated
heat transfer through the composite. Thus, the ratio of separate
materials in filler and the ratio of filler and polymer must both be
considered for hybrid-filler polymer composites. Multiple fillers mixed
into a polymer can also exhibit a synergistic effect and improve thermal
conductivity. Lee, et. al. found that a filler mixture of BN and alumina
fiber slightly improved thermal conductivity over a BN single-filler by
as much as 0.2 W/mK, despite the low thermal conductivity of alumina
over BN.

\subsection{Effect of mixing method on thermal conductivity}

Zhou, et. al. found that the mixing method of a polymer composite
significantly affects thermal conductivity. HDPE-BN samples were
prepared by both a powder mixing method and melt mixing method. While
little to no change in thermal conductivity was measured at low filler
content, at higher concentrations of BN the powder-mixed sample measured
a higher thermal conductivity, up to 0.2 W/mK more than the melt-mixed
sample. SEM photography showed that the powder-mixed composite contained
a more even, networked dispersion of BN within the polymer matrix. This
leads to a greater surface area of BN around HDPE particles and more
conductive pathways for heat to flow through the material. Conversely,
the BN melt-mixed composite formed large globules, which would not be
conducive to a higher thermal conductivity.

\subsection{Effect of surface treatment on thermal conductivity}

Application of a surface treatment to polymer composites has been shown
to improve even dispersion of filler within the polymer and increase
thermal conductivity. Lee, et. al. utilized a titanate coupling agent on
multiple single and hybrid-filler composites. All treated samples
displayed an enhanced thermal conductivity of about 0.14 W/mK compared
to their untreated counterparts.\cite{Lee-2006-727}  Applying a surface treatment
ideally decreases the amount of phonon deflection at the boundary
between polymer and filler, which allows for more heat transfer through
the material.

\section{Carbon-Polymer Hybrids}

Polymers mixed with various forms of elemental carbon have shown a
significant increase in thermal conductivity in experimental study when
compared to pure polymers. Grundler et. al. studied a sample of
polyamide 6 mixed with graphite at concentrations ranging from 10 to 74
wt\%. Results showed an exponential rise in thermal conductivity with an
increase in filler content, with a maximum reading at between 25 W/mK
and 30 W/mK, which represents a dramatic rise in magnitude over pure
polymers' common thermal conductivities of under 1 W/mK.\cite{Grundler-2016}  Chen
et. al. utilized an epoxy composite pressed into mats of vapor-grown
carbon fiber, at filler volume percents ranging from 14 to 56\%. Even at
relatively low to intermediate filler content, the thermal conductivity
of the composite was measured to be about 200 to over 600 W/mK in the
x-axis, and 4-35 W/mK in the y-axis. Both measurements were taken to
account for samples possessing a 1-dimensional and 2-dimensional carbon
fiber configuration.\cite{Chen-2002-359}  Wong, et. al. mixed a thermotropic liquid
crystal polymer with carbon black at low volume fraction, between 0 and
10\% CB. Thermal conductivity was measured along all three dimensions.
Along the y-axis, thermal conductivity remained constant, while it
decreased and increased along the x and z-axes, respectively. Thermal
conductivity as a function of filler concentration converged along all
dimensions to about 0.45 W/mK. Carbon black was found to contribute
significantly less to thermal conductivity of the polymer than
electrical conductivity, which was also measured.\cite{Wong-2001-1549}

\subsection{Effect of temperature on thermal conductivity}

Temperature has been shown to slightly affect thermal conductivity of
polymer composites due to thermal expansion of the material. Grundler,
et. al. mixed graphite with polypropylene and measured thermal
conductivity as a function of temperature at filler concentrations
ranging from 10 to 70 wt\%. Results indicated a slight decrease of
2.48\% in thermal conductivity from 30 degrees to 110 degrees Celsius.\cite{Grundler-2016}  Polypropylene, a thermoplastic, displays significant thermal
expansion at the measured temperatures, and thus Grundler, et. al.
theorize that lengthening the distance between polymer molecules impedes
efficient heat flow through the material.\cite{Grundler-2016}

\subsection{Effect of filler orientation on thermal conductivity}

The spatial orientation of filler within a polymer affects the number of
networks through which the material can conduct heat. Grundler, et. al.
found that injection molding of graphite flakes into a polymer to form a
carbon-based composite created unique areas of uniform flake direction.
SEM photography indicated a vertical filler alignment in the middle of
the sample and horizontal alignment on the surface and bottom of the
sample. Thermal conductivity measurements by region found that thermal
conductivity vertically through the entirety of the composite is
significantly higher than that measured vertically or horizontally
through the core of the sample.\cite{Grundler-2016}  Thus, to maintain uniform
thermal conductivity, uniform orientation of filler within the polymer
is of distinct importance. Chen, et. al. noted the probability of
defects within the VGCF filler. Fibers within the carbon fiber mats may
be misaligned, leading to an inflated measurement of thermal
conductivity along the x-axis. This may explain the relatively large
thermal conductivity of 695 W/mK found along this dimension.\cite{Chen-2002-359}

\section{Conclusion}

Experimental results show the significant possibility of utilizing
highly conductive fillers such as metals, ceramics, and carbon in
conjunction with a relatively non-conductive polymer for thermal and
electronic applications. Furthermore, these materials possess the
advantage of being thermally conductive and electrically insulating.
Observations from research into polymer composites indicate that the
geometry, orientation, and dispersion of filler particles within the
polymer matrix are most essential in determining the ultimate thermal
conductivity of the material. Factors such as surface modification,
temperature, and production method may also be considered. Research must
continue into improving the thermal conductivity of polymers using the
above factors. Nevertheless, polymer composites appear promising in
terms of replacing metallic components for heat controls and
electronics.

\bibliography{./TermPaper.bib}

\end{multicols}

\begin{table}[h]
  \centering
  \caption{\textbf{Random Appendix}}
  \label{tab:fe}
  \vspace{6pt}
  \renewcommand{\arraystretch}{1.2}
  \begin{tabular}{|l|r|}
    \hline
    One & Two \\
    \hline
    Three & Fourteen \\
    \hline
  \end{tabular}
\end{table}

\section{Final Thoughts}

Using \LaTeX seems to be a much better option when doing simple word processing. It has many robust options for formatting text and text position on the page, and in a much more intuitive way than Microsoft Word. However, with Word I find it easier to quickly insert a table or figure, and it is a lot easier to move these objects around the page, give options for formatting the text around an image or table, and see real-time results. On that note as well, \LaTeX is more akin to coding, where changes are slow and incremental and you need to compile each time. Other options which I think we should look into or see demonstrated are the ability to convert to doc/docx, fixing spacing issues with multicol (can be seen if you compile this code!), and how to insert non-pdf figures.
\end{document}

%%% Local Variables:
%%% mode: latex
%%% TeX-master: t
%%% End:
