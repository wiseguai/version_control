\PassOptionsToPackage{unicode=true}{hyperref} % options for packages loaded elsewhere
\PassOptionsToPackage{hyphens}{url}
%
\documentclass[]{article}
\usepackage{lmodern}
\usepackage{amssymb,amsmath}
\usepackage{ifxetex,ifluatex}
\usepackage{fixltx2e} % provides \textsubscript
\ifnum 0\ifxetex 1\fi\ifluatex 1\fi=0 % if pdftex
  \usepackage[T1]{fontenc}
  \usepackage[utf8]{inputenc}
  \usepackage{textcomp} % provides euro and other symbols
\else % if luatex or xelatex
  \usepackage{unicode-math}
  \defaultfontfeatures{Ligatures=TeX,Scale=MatchLowercase}
\fi
% use upquote if available, for straight quotes in verbatim environments
\IfFileExists{upquote.sty}{\usepackage{upquote}}{}
% use microtype if available
\IfFileExists{microtype.sty}{%
\usepackage[]{microtype}
\UseMicrotypeSet[protrusion]{basicmath} % disable protrusion for tt fonts
}{}
\IfFileExists{parskip.sty}{%
\usepackage{parskip}
}{% else
\setlength{\parindent}{0pt}
\setlength{\parskip}{6pt plus 2pt minus 1pt}
}
\usepackage{hyperref}
\hypersetup{
            pdfborder={0 0 0},
            breaklinks=true}
\urlstyle{same}  % don't use monospace font for urls
\usepackage{longtable,booktabs}
% Fix footnotes in tables (requires footnote package)
\IfFileExists{footnote.sty}{\usepackage{footnote}\makesavenoteenv{longtable}}{}
\usepackage{graphicx,grffile}
\makeatletter
\def\maxwidth{\ifdim\Gin@nat@width>\linewidth\linewidth\else\Gin@nat@width\fi}
\def\maxheight{\ifdim\Gin@nat@height>\textheight\textheight\else\Gin@nat@height\fi}
\makeatother
% Scale images if necessary, so that they will not overflow the page
% margins by default, and it is still possible to overwrite the defaults
% using explicit options in \includegraphics[width, height, ...]{}
\setkeys{Gin}{width=\maxwidth,height=\maxheight,keepaspectratio}
\setlength{\emergencystretch}{3em}  % prevent overfull lines
\providecommand{\tightlist}{%
  \setlength{\itemsep}{0pt}\setlength{\parskip}{0pt}}
\setcounter{secnumdepth}{0}
% Redefines (sub)paragraphs to behave more like sections
\ifx\paragraph\undefined\else
\let\oldparagraph\paragraph
\renewcommand{\paragraph}[1]{\oldparagraph{#1}\mbox{}}
\fi
\ifx\subparagraph\undefined\else
\let\oldsubparagraph\subparagraph
\renewcommand{\subparagraph}[1]{\oldsubparagraph{#1}\mbox{}}
\fi

% set default figure placement to htbp
\makeatletter
\def\fps@figure{htbp}
\makeatother


\date{}

\begin{document}

\textbf{Phase Transformations and Hardenability of Steel}

Ian Wise, EMSE 330

\textbf{Abstract}

Steel is a universally-recognized structural material made via adding
small concentrations of carbon to iron. Carbon interstitially combines
with the iron to form a strong solid solution that exhibits good
mechanical properties, including hardness, for many engineering
applications. Additional transition elements can be added to the iron as
well to produce stronger steels with good hardenability, or the ability
to form strong Fe-C phases upon quenching. This experiment observes the
differences in microstructure and hardness among two plain-carbon and
two alloy steels with different heat treatment methods. Optical
microscopy was used to examine the microstructure, and hardness test
were conducted. Larger bars of each steel were subjected to a Jominy
test followed by hardness testing to compare the hardenability of the
steels.

\textbf{Introduction}

\includegraphics[width=3in,height=2.425in]{media/image1.png} As one of
the most commonplace structural materials, steel and steel alloys remain
integral to a significant majority of modern societies. Therefore, it is
necessary to understand the properties of steel and the effects of heat
and alloying elements on those properties in order to efficiently
utilize steel in modern applications. Many mechanical properties of
steel are directly connected to the microstructure of the material,
which can change as a function of heat treatment temperature and as new
elements are added to the iron matrix. The most important of these
elements, carbon, is widely understood in terms of its interactions with
iron, as Figure 1 (the Fe-C phase diagram) displays.

\includegraphics[width=2.25736in,height=3.24965in]{media/image2.png}For
the purposes of this experiment, we will consider the portion of Figure
1 with a C-content of below 0.76wt\%. Steels of these compositions are
known as hypoeutectoid steels, as their carbon compositions lie below
the eutectoid composition of 0.76wt\% C. All samples in this experiment
are hypoeutectoid steels. Below the eutectoid temperature of 727°C, the
austenite will transform into two distinct phases: a proeutectoid
ferrite (α) matrix, which is soft and ductile, and a pearlite phase.
Pearlite consists of alternating phases of ferrite and cementite
(Fe\textsubscript{3}C), a brittle and hard iron-carbon compound, in a
lamella formation. The lamellae structure creates the shortest path
through which carbon can diffuse from the ferrite matrix and form
cementite layers. Figure 2 illustrates a typical cooling sequence for a
hypoeutectoid steel, along with an actual photomicrograph depicting the
lamellar grain structure of pearlite.

The Fe-C phase diagram considers only continuous equilibrium cooling, in
which enough time is given for complete diffusion. In realistic
scenarios, most steels undergo non-equilibrium cooling. Cooling
generally follows an exponential decline due to the lowering discrepancy
in energy between the material and its surroundings equating to less of
a driving force towards heat transfer. The cooling behavior of materials
can be modeled using the Johnson, Mehl, Avrami, Kormogorov (JMAK)
equation:

\[f = 1 - e^{- kt^{n}}\]

where t is time, and n and k are time-independent constants.

Non-equilibrium cooling yields both a shift in the phase transformation
temperatures (indicated as red lines separating phases in Figures 1 and
2) and the existence of non-equilibrium phases at room temperature. In
an engineering setting, non-equilibrium cooling in utilized to control
the microstructure of the material and its associated properties.

Isothermal time-temperature transformation (TTT) diagrams illustrate the
transformation of a material from one phase to another when held at a
certain temperature. Solid curves generally correlate to the beginning
and end of a phase transformation, and dashed lines in-between
correspond to 50\% transformation. TTT diagrams also vary depending on
the composition of the material.

\includegraphics[width=2.925in,height=3.17479in]{media/image3.png}
\includegraphics[width=2.95329in,height=3.21864in]{media/image4.png}

\emph{Figure 3: Time-temperature transformation (TTT) diagrams for a
typical hypoeutectoid steel of a) plain-carbon and b) alloyed
composition {[}1{]}}

For steels and steel alloys, TTT diagrams indicate a number of phases
which can form beyond those given from an equilibrium phase diagram. Of
these phases, three are of note to this experiment:

\begin{itemize}
\item
  Bainite is an analog of pearlite that forms at lower temperatures, or
  below the pearlite ``nose'' as seen in Figure 3. This mixture of
  ferrite and cementite is different in that the cementite forms a more
  needle-like, elongated shape in the ferrite matrix. Bainite is
  generally harder than pearlite and has good ductility. This phase can
  also not be directly transformed into pearlite via a higher
  temperature.
\item
  High carbon diffusion rates at temperatures just below the eutectoid
  temperature result in the cementite forming spheres in the ferrite
  matrix to reduce the surface energy between the two phases; this
  configuration is known as spheroidite. Given that dislocations can
  easily move through the ferrite matrix, this microstructure is very
  soft and weak.
\item
  If a steel is quenched, diffusion is not able to occur, and the carbon
  that slips out of solid solution will form interstitial impurities in
  the crystal structure of the iron. This distorts the crystal lattice
  from face-centered cubic to body-centered tetragonal and forms
  martensite, a supersaturated solid solution which resembles plates or
  needle-like shapes.
\end{itemize}

Martensite is very strong, but often too brittle for engineering
applications. Thus, martensite will often be tempered at elevated
temperatures well below the eutectoid temperature. Given that martensite
is a non-equilibrium phase, carbon will quickly diffuse into small
spheres in the material. This allows the steel to maintain its high
hardness while increasing its toughness and ductility. Figure 4
illustrates typical microstructures for the three non-equilibrium phases
discussed above.

\includegraphics[width=1.59278in,height=2.24278in]{media/image5.png}
\includegraphics[width=1.81503in,height=2.22888in]{media/image6.png}
\includegraphics[width=1.81in,height=2.24in]{media/image7.png}

\emph{Figure 4: Typical non-equilibrium microstructures of hypoeutectoid
carbon steels: a) bainite, b) spheroidite, and c) martensite {[}1{]}}

Continuous-cooling transformation (CCT) diagrams are also used to
predict the microstructure of a material. These diagrams tend to be more
realistic than TTT diagrams, as they shift the required time of a
reaction forward as a function of the cooling rate. For steels, CCT
diagrams are significant in that they often show that for low-carbon
(\textless{}0.25wt\%C) steels, martensite may not form due to its
required cooling rate being impractically high. Figure 5 shows typical
CCT diagrams for hypoeutectoid steels.

\includegraphics[width=2.87141in,height=3.27361in]{media/image8.png}
\includegraphics[width=2.92in,height=3.26in]{media/image9.png}

\emph{Figure 5: Continuous cooling transformation (CCT) diagrams for a
typical hypoeutectoid steel of a) plain-carbon and b) alloyed
composition {[}1{]}}

For alloyed steels, the addition of elements shifts the TTT curve to the
right, which increases the hardenability of the steel, or its ability to
form martensite in the steel interior. To measure hardenability, a
Jominy end-quench test can be conducted. One end of a steel bar is
quenched, and after cooling the hardness is measured horizontally along
the bar's surface. A steel with high hardenability will display higher
hardness further away from the quenched end, indicating a more complete
martensitic transformation.

\textbf{Materials and Methods}

Four steel compositions were selected for testing. The chemical
compositions of these samples are listed in Table 1:

\begin{longtable}[]{@{}llllll@{}}
\toprule
\textbf{Label} & \textbf{Designation} & \textbf{C (wt\%)} & \textbf{Cr
(wt\%)} & \textbf{Mo (wt\%)} & \textbf{Ni (wt\%)}\tabularnewline
\midrule
\endhead
A & 1018 & 0.18 & 0 & 0 & 0\tabularnewline
B & 1045 & 0.45 & 0 & 0 & 0\tabularnewline
C & 4140 & 0.40 & 0.80-1.10 & 0.15-0.25 & 0\tabularnewline
D & 4340 & 0.40 & 0.40-0.90 & 0.20-0.30 & 1.65-2.00\tabularnewline
\bottomrule
\end{longtable}

\emph{Table 1: Chemical compositions of the four steels tested}

Of each composition, 8 cylindrical samples of 1cm in diameter and 1cm in
height were received, as was 1 Jominy test bar of approximately 10cm in
length.

The 8 small samples were heated to 900°C for 1 hour in a furnace. The
following lists the subsequent cooling/annealing schedule for each
sample:

\begin{enumerate}
\def\labelenumi{\arabic{enumi}.}
\item
  Water quenched.
\item
  Water quenched. Heated to 200°C for 1 hour.
\item
  Water quenched. Heated to 400°C for 1 hour.
\item
  Water quenched. Heated to 700°C for 1 hour.
\item
  Water quenched. Heated to 700°C for 24 hours.
\item
  Air cooled.
\item
  Quenched in a Pb bath (350°C) for 3 hours.
\item
  Furnace cooled.
\end{enumerate}

The Jominy test bars were heated to 900°C for 3 hours in a furnace,
followed by a quenching on one end of the bar by a continuous stream of
water, for approximately 10 minutes. The bars were then air cooled to
room temperature. All samples were ground, polished, and etched for
optical microscopy according to the following schedule:

\begin{enumerate}
\def\labelenumi{\arabic{enumi}.}
\item
  240 grit SiC paper
\item
  600 grit SiC paper
\item
  800 grit SiC paper
\item
  1200 grit SiC paper
\item
  1μm diamond polish
\item
  0.05μm alumina polish
\item
  Nital (5\%) etch
\end{enumerate}

The smaller samples were polished on one cylindrical face. The Jominy
bars were polished length-wise on one side to illustrate the change in
microstructure as a function of distance from the quenched end of the
bars.

All samples were imaged using the Keyence optical microscope. Following
microscopy, all samples were hardness tested using a Rockwell C-scale
hardness tester. The small samples were tested four times each, and the
average was taken to be the hardness of the sample. The Jominy bars were
hardness tested at specific length increments from the quenched end to
demonstrate the differences in hardenability between each composition.

\textbf{Results and Discussion}

The following figures display TTT and CCT diagrams for the four steel
compositions tested or those similar:

\includegraphics[width=2.81745in,height=2.75in]{media/image10.png}\includegraphics[width=2.83014in,height=2.75in]{media/image11.png}

\includegraphics[width=3.25in,height=2.61277in]{media/image12.png}\includegraphics[width=2.72083in,height=2.60903in]{media/image13.png}

\emph{Figure 6: TTT diagrams for a) 1018, b) 1045, c) 4140, and d) 4340
steel {[}2{]}}

\includegraphics[width=2.07994in,height=2.69792in]{media/image14.png}\includegraphics[width=2.02438in,height=2.69182in]{media/image15.png}\includegraphics[width=2.09044in,height=2.74325in]{media/image16.png}

\emph{Figure 7: CCT diagrams for a) low-carbon, b) medium-carbon, and c)
Ni-Cr-Mo alloyed steel.}

Figure 7A corresponds closely to the 1018 steel samples (group A). The
small blue-shaded area on the left of the graph indicates that a very
large cooling rate is required to form martensite without forming
bainite first, and this may not be feasible for these samples even with
a water quench. At lower cooling rates, such as during an air or furnace
cooling, low-carbon steels will ideally form pearlite. Figure 7A also
shows that hardness for these samples should rank lowest among all
compositions tested, due to the low carbon concentration promoting the
growth of soft ferrite.

Figure 7B fits the 1045 steel samples (group B). Again, the figure
points to a large cooling rate required to form martensite; however, the
scenario is more realistic given that the martensite area extends
further to the right. At a lower cooling rate or less intense quench,
the 1045 samples should ideally directly form bainite, as opposed to the
1018 samples which, according to Figure 7A, may reach the ferrite area
first. A very low cooling rate on Figure 7B will also promote the growth
of pearlite. For hardenability, while at small sample sizes (which
correspond to higher cooling rates), the hardness of 1045 samples
increase exponentially, for the purposes of this experiment the hardness
should be relatively equal to that of the 1018 samples.

Figure 7C corresponds most closely to the 4340 steel samples (group D),
as it is for a Ni-Cr-Mo alloy; the CCT diagram can be approximated for
the 4140 steel samples as well (group C). A larger area corresponding to
martensite points to the fact that rapid quenching should induce a
martensitic transformation in alloyed steels. Furthermore, the very
small pearlite area means that a very slow cooling curve is required to
form pearlite; for most slow-cooling scenarios, bainite is likely to
form in these samples.

\emph{\textbf{Optical Microscopy}}

All small sample images were taken at 500x magnification. In each table,
the letter on each figure corresponds to the sample groups and
compositions indicated in Table 1.

\emph{1. Water quenched.}

\begin{longtable}[]{@{}ll@{}}
\toprule
\includegraphics[width=3.26559in,height=2.45004in]{media/image17.jpeg} &
\includegraphics[width=3.26553in,height=2.45in]{media/image18.jpeg}\tabularnewline
\midrule
\endhead
\includegraphics[width=3.26553in,height=2.45in]{media/image19.jpeg} &
\includegraphics[width=3.26553in,height=2.45in]{media/image20.jpeg}\tabularnewline
\bottomrule
\end{longtable}

\emph{Figure 8: As-quenched a) 1018, b) 1045, c) 4140, and d) 4340 steel
samples}

The quenched, unannealed steel samples should ideally have transformed
into martensite due to the diffusionless nature of such rapid cooling.
Figure 8A shows fewer dark formations than the other samples, which may
indicate in incomplete martensitic transformation due to the low-carbon
composition. Figure 8B, also plain-carbon steel, does not show a
microstructure typical of martensite; instead, the cementite phase
appears to have segregated to the grain boundaries of the white ferrite
matrix. The darker surface color and needle-like grains in Figures 8C
and 8D, both alloy steels, indicate a martensitic transformation.

\emph{2. Water quenched. Heated to 200°C for 1 hour.}

\begin{longtable}[]{@{}ll@{}}
\toprule
\includegraphics[width=3.26667in,height=2.45in]{media/image21.jpeg} &
\includegraphics[width=3.26667in,height=2.45in]{media/image22.jpeg}\tabularnewline
\midrule
\endhead
\includegraphics[width=3.26667in,height=2.45in]{media/image23.jpeg} &
\includegraphics[width=3.26667in,height=2.45in]{media/image24.jpeg}\tabularnewline
\bottomrule
\end{longtable}

\emph{Figure 9: Quenched a) 1018, b) 1045, c) 4140, and d) 4340 steel
samples annealed at low temperature}

After annealing at low temperature, our samples should ideally display
tempered martensite -- a microstructure similar to spheroidite, but with
precipitated cementite spheres much smaller in size. Figure 9A, the 1018
steel, appears darker than in the unannealed sample. Upon higher
magnification, one may be able to see the array of precipitated
cementite. Conversely, Figure 9C is lighter than in the unannealed 4140
steel sample, which may indicate a more thorough decomposition of the
martensite into a stable ferrite/cementite microstructure. However,
large dark conglomerations are still present, indicating that not all of
the martensite has been transformed. In Figure 9B, it is difficult to
assess whether the dark area is high in martensite or the result of
uneven etching. In Figure 9D, the surface appears underpolished and
displays pitting corrosion. Areas of high carbon concentration cannot be
identified.

\emph{3. Water quenched. Heated to 400°C for 1 hour.}

\begin{longtable}[]{@{}ll@{}}
\toprule
\includegraphics[width=3.26667in,height=2.45in]{media/image25.jpeg} &
\includegraphics[width=3.26667in,height=2.45in]{media/image26.jpeg}\tabularnewline
\midrule
\endhead
\includegraphics[width=3.26553in,height=2.44915in]{media/image27.jpeg} &
\includegraphics[width=3.26667in,height=2.45in]{media/image28.jpeg}\tabularnewline
\bottomrule
\end{longtable}

\emph{Figure 10: Quenched a) 1018, b) 1045, c) 4140, and d) 4340 steel
samples annealed at medium temperature}

At this higher annealing temperature, the microstructure of these
samples should show a more complete transformation from martensite into
stable ferrite and cementite. The lighter color of Figure 10A indicates
that precipitated cementite has distributed more uniformly through the
sample. Figure 10B also shows a lack of martensite across the sample
surface. Some small pits can also be seen along the surface. In the 4140
sample corresponding to Figure 10C, the microstructure shows less carbon
agglomerations than in the samples annealed at a lower temperature,
which may indicate a more uniform dispersion of cementite in the ferrite
matrix. The 4340 sample corresponding to Figure 10D appears too
unpolished to clearly make out the microstructure; however, one can see
some darker patches, which could indicate an area of high carbon
concentration or could indicate surface grime.

\emph{4. Water quenched. Heated to 700°C for 1 hour.}

\begin{longtable}[]{@{}ll@{}}
\toprule
\includegraphics[width=3.26667in,height=2.45in]{media/image29.jpeg} &
\includegraphics[width=3.26667in,height=2.45in]{media/image30.jpeg}\tabularnewline
\midrule
\endhead
\includegraphics[width=3.26553in,height=2.44915in]{media/image31.jpeg} &
\includegraphics[width=3.26667in,height=2.45in]{media/image32.jpeg}\tabularnewline
\bottomrule
\end{longtable}

\emph{Figure 11: Quenched a) 1018, b) 1045, c) 4140, and d) 4340 steel
samples annealed at high temperature}

Annealing at a temperature close to the eutectoid temperature for 1 hour
should ideally yield a microstructure resembling bainite in these
samples according to their TTT diagrams. Figure 11B indicates that this
transformation has taken place, as one can see the elongated cementite
framework running through and around the ferrite matrix. In Figures 11A
and 11D, however, the microstructure cannot be determined due to
underpolishing/underetching and severe overetching/pitting corrosion,
respectively. In Figure 11C, pitting corrosion is also present, but
slight dark areas indicate a possible bainitic microstructure, which may
be able to be confirmed via higher magnification or more thorough
polishing.

\emph{5. Water quenched. Heated to 700°C for 24 hours.}

\begin{longtable}[]{@{}ll@{}}
\toprule
\includegraphics[width=3.26667in,height=2.45in]{media/image33.jpeg} &
\includegraphics[width=3.26667in,height=2.45in]{media/image34.jpeg}\tabularnewline
\midrule
\endhead
\includegraphics[width=3.26553in,height=2.44915in]{media/image35.jpeg} &
\includegraphics[width=3.26667in,height=2.45in]{media/image36.jpeg}\tabularnewline
\bottomrule
\end{longtable}

\emph{Figure 12: Quenched a) 1018, b) 1045, c) 4140, and d) 4340 steel
samples annealed at high temperature for a long time}

A long annealing time at just below the eutectoid temperature should
yield spheroidite, which can be identified by black dots of cementite
large enough to be individually identified at this magnification. In
Figures 12B and 12D, one can make out a sea of these dots within the
white grains of ferrite, and so these samples show that the
transformation into spheroidite was successful. In the 1045 sample,
smaller grains of pearlite have also formed, as indicated by the
lamellar structure. Small black patches can also be seen in Figure 12A;
these may indicate formations of spheroidite or formations of pearlite.
The microstructure is less clear in Figure 12C, although the surface is
more uniform than in the sample heated for 1 hour at the same
temperature.

\emph{6. Air cooled.}

\begin{longtable}[]{@{}ll@{}}
\toprule
\includegraphics[width=3.26667in,height=2.45in]{media/image37.jpeg} &
\includegraphics[width=3.26667in,height=2.45in]{media/image38.jpeg}\tabularnewline
\midrule
\endhead
\includegraphics[width=3.26553in,height=2.44915in]{media/image39.jpeg} &
\includegraphics[width=3.26667in,height=2.45in]{media/image40.jpeg}\tabularnewline
\bottomrule
\end{longtable}

\emph{Figure 13: As-air cooled a) 1018, b) 1045, c) 4140, and d) 4340
steel samples}

Layers of pearlite should ideally form in these samples due to the
relatively low cooling rate in air. In Figure 13B, the microstructure
appears quite textured, indicating that pearlite may have formed.
However, it is difficult to tell if the darker patterning resembles the
lamellar structure typical of pearlite. Similarly, the microstructure is
difficult to discern in the other samples. In Figure 13A, beyond the
corrosion of the surface one can make out many dark blotches, which at
lower magnifications can be indicative of pearlite formation. Figure 13C
appears to be overetched, which obscures the microstructure. In Figure
13D, the darker formations may be due to overetching or may be pearlite
formations, but it is difficult to discern between the two scenarios.

\emph{7. Quenched in a Pb bath (350°C) for 3 hours.}

\begin{longtable}[]{@{}ll@{}}
\toprule
\includegraphics[width=3.26667in,height=2.45in]{media/image41.jpeg} &
\includegraphics[width=3.26667in,height=2.45in]{media/image42.jpeg}\tabularnewline
\midrule
\endhead
\includegraphics[width=3.26553in,height=2.44915in]{media/image43.jpeg} &
\includegraphics[width=3.26667in,height=2.45in]{media/image44.jpeg}\tabularnewline
\bottomrule
\end{longtable}

\emph{Figure 14: As-lead quenched a) 1018, b) 1045, c) 4140, and d) 4340
steel samples}

A fast quench from 900°C should form some martensite in these samples,
and the long time held at this temperature may form a bainite
microstructure as well. Figure 14D appears too underpolished to discern
a bainite microstructure, and the needle-like formations indicative of
martensite area also not present. In Figure 14C, the similarity of the
color and texture of the surface to samples annealed at high
temperatures may indicate that martensite did not form. Bainite or
martensite appears to be present in Figure 14A, as shown by the sea of
darker formations. Unfortunately, the microstructure in Figure 14B
cannot be determined, as either the sample is heavily etched or the
contrast obscures the grain boundaries.

\emph{8. Furnace cooled.}

\begin{longtable}[]{@{}ll@{}}
\toprule
\includegraphics[width=3.26667in,height=2.45in]{media/image45.jpeg} &
\includegraphics[width=3.26667in,height=2.45in]{media/image46.jpeg}\tabularnewline
\midrule
\endhead
\includegraphics[width=3.26553in,height=2.44915in]{media/image47.jpeg} &
\includegraphics[width=3.26667in,height=2.45in]{media/image48.png}\tabularnewline
\bottomrule
\end{longtable}

\emph{Figure 15: As-furnace cooled a) 1018, b) 1045, c) 4140, and d)
4340 steel samples}

Having the slowest cooling rate, pearlite should be the dominant phase
transformation in these samples. Figure 15A has a very dark surface, and
this may be due to overetching which obscures the true microstructure.
Pearlite formations can be seen in Figures 15B and 15D, as indicated by
formations of dark patches speckled with light spots; these speckles at
a low magnification can indicate a lamellar structure. In Figure 15C,
two distinct areas have been shown, one light and one dark. This may be
due to an uneven polish or etch, but one can clearly make out dark
patches throughout the surface which may indicate pearlite formation.

Jominy bars were imaged near the middle of each bar, to illustrate the
difference in the extent of martensite formation from the quenched end
for each composition. Images for samples A and C were taken at 500x
magnification, and for samples B and D 1000x magnification.

\begin{longtable}[]{@{}ll@{}}
\toprule
\includegraphics[width=3.26667in,height=2.45in]{media/image49.jpeg} &
\includegraphics[width=3.26667in,height=2.45in]{media/image50.jpeg}\tabularnewline
\midrule
\endhead
\includegraphics[width=3.26667in,height=2.45in]{media/image51.jpeg} &
\includegraphics[width=3.26667in,height=2.45in]{media/image52.jpeg}\tabularnewline
\bottomrule
\end{longtable}

\emph{Figure 16: Jominy bars of a) 1018, b) 1045, c) 4140, and d) 4340
steel samples}

Ideally, the plain-carbon steels should appear lighter due to a lower
martensite concentration. Figure 16D, an alloy steel, illustrates
martensitic transformation upon quenching. The needle-like, dark
concentrations are martensite. Conversely, Figure 16C is much lighter
and has an orange hue, which may indicate a majority ferrite
concentration. The lack of martensite may be due to an insufficient
quenching time. An uneven surface in Figure 16B makes it difficult to
assess the microstructure. The dark blotches on the surface are not
typical of martensite, but do indicate cementite agglomeration, which is
more typical of a medium-carbon steel. In Figure 16A, the lines of dark
areas may be due to polishing or surface imperfections.

\emph{\textbf{Hardness}}

Figure 17 lists hardness data for each small sample. Ideally, one should
see the lowest average hardness in the rightmost grouping, as a long
annealing time should form spheroidite, a soft phase of steel. Likewise,
the as-quenched samples should form martensite upon quenching, and thus
should exhibit the highest hardness. Furthermore, the 1018 carbon steel
should ideally be softer than the other samples due to the lower carbon
content promoting the growth of soft ferrite.

\includegraphics[width=3.51389in,height=2.5in]{media/image53.png}The
hardness data appears to agree with ideality. The annealed samples
generally trend downward in hardness as the temperature and time of
annealing increases. This trend makes sense due to more decomposition of
the harder martensite phase with higher temperature annealing and longer
annealing times. Samples which experienced lower initial cooling rates,
namely the air-cooled furnace-cooled samples, generally display higher
hardness than the high-temperature annealed samples, but a lower
hardness than the low or mid-temperature annealed samples. From this
data and using the CCT diagrams of these steels, one could draw a
conclusion as to at how high a temperature quenched steels must be
tempered to overcome their initial martensite formation, to a point
where ductility is improved (hardness is lowered) versus via convection
cooling.

In most sample groups, the medium-carbon steels exhibit a much higher
hardness than the 1018 (A) group. This trend is expected, as a higher
carbon concentration equates to a more thorough growth of cementite in
the steel, which with its brittleness will contribute to a high
hardness. Some samples were also initially measured using a Rockwell B
scale and later converted to Rockwell C. However, samples A and B for
the furnace-cooled group were not able to be converted.

\includegraphics[width=3.86667in,height=2.49028in]{media/image54.png}\emph{\textbf{Hardenability}}

Figure 18 illustrates the results of the Jominy bar hardenability test.
Unfortunately, hardness measurements for each bar were taken starting
from the cap of the bars towards the quenched end. Thus, the trend in
hardness for ach sample is slightly increasing as with each measurement
a more martensite-rich area is reached. Samples C and D, the alloy
steels, have a higher hardness than Sample B, the 1045 plain-carbon
steel. This trend fits well with the CCT diagrams for alloy steel and
medium-carbon steel in Figure 7. The addition of alloying elements into
the iron induces precipitation hardening and strengthening mechanisms,
which contribute to the hardness of the steel. As can be seen in the TTT
diagrams in Figure 6, the hardenability of alloyed steels should
increase as well. The shifting of the bainite/pearlite nose to the right
indicates that more martensite will form in these samples, and so the
hardness should remain high for a longer portion of the sample. However,
since the slopes of the data lines in Samples B, C, and D are all
similar, increased hardenability in the alloyed steels was not observed.
Sample A was tested under the Rockwell B scale and could not be
converted to Rockwell C to be compared to the other samples; however,
given the average hardness measured, if hardness measurements were
initially taken under a C scale, Sample A should roughly align with
Sample B. Jominy bar hardness measurements were not able to be
remeasured due to the existing hardness measurements artificially
hardening the sample surface by forming compressive stress
concentrations.

\textbf{Summary}

Overall, the experiments conducted to observe the effects of carbon
content and annealing temperature and time on steel fit well with
theory. Samples with lower carbon concentrations displayed less
martensite formation in the surface microstructure and a lower hardness.
Furthermore, samples annealed at higher temperatures and for a longer
period of time exhibited lower hardness and a higher percentage of
ferrite. The Jominy hardenability test somewhat fit theoretical
predictions. Higher hardness among the alloyed steel samples as compared
to plain-carbon steels was expected. However, a higher hardenability
among the alloyed steels was not observed. Improper hardness
calculations also prevented accurate comparison of hardenability among
some samples.

Quench-hardened steels often exhibit very high hardness and strength due
to martensite formation, but at the cost of ductility required for many
structural applications. However, there are still uses and research
conducted for quench-hardened steels. In the automotive and machinery
industry, quench-hardened carburized steels are often integrated into
small parts such as gears and drills, which must withstand very high
stresses and temperatures {[}3{]}. The automotive industry has also
begun to utilize these steels in the body of vehicles (e.g. door beams,
bumpers, cross bars) to capitalize on low weight and structural
performance {[}4{]}. Quench-hardened steels are commonplace in the oil
\& gas and tool-making industries; quenched alloy steels such as MnCrB,
NiCrSi, and NiCrMoV have been studied to determine their mechanical
properties for use in these applications {[}5{]}. Martensitic steels
have been widely used in drill piping, casing, and tubing in oil
drilling, and have been researched to improve their resistance to stress
corrosion from sulfide contamination, which reduces their effectiveness
{[}6{]}. Finally, both ferrite and martensite exhibit a relatively high
magnetic permeability; thus, martensitic stainless steels (containing
Cr) can also find applications for use in solenoid cores and permanent
magnets {[}7{]}. As more research into the properties of martensite and
martensitic steels is done, it is likely that the opportunities for
their use in industry and manufacturing will expand.

\textbf{References}

{[}1{]}. Callister, W. D.; Rethwisch, D. G. \emph{Fundamentals of
Materials Science and Engineering: An Integrated Approach},
5\textsuperscript{th} ed. John Wiley \& Sons, Inc., \textbf{2015}.

{[}2{]}. Kappes, M.; Iannuzzi, M.; Bebak, R. B.; Carranza, R. M. Sulfide
stress cracking of nickel containing low-alloy steels. \emph{Corrosion
Reviews}, \textbf{2014}, 32, pp. 101-128.

{[}3{]}. Su, S.-r.; Song, R.-b.; Chen, C.; Wang, J.-y.; Zhang, Y.-c. The
novel process of spheroidizing-critical annealing used to optimize the
properties of carburized steel and its effect on hardening mechanism of
quenching and tempering. \emph{Materials Science and Engineering: A},
\textbf{2019}, 765, 138322.

{[}4{]}. Horvath, C.D. ``Advanced steels for lightweight automotive
structures.'' \emph{Materials, Design and Manufacturing for Lightweight
Vehicles}, Woodhead Publishing, \textbf{2010}, pp. 35-78.

{[}5{]}. Abbasi, E.; Luo, Q.; Owens, D. A comparison of microstructure
and mechanical properties of low-alloy-medium-carbon steels after
quench-hardening. \emph{Materials Science and Engineering: A},
\textbf{2018}, 725, pp. 65-75.

{[}6{]}. Wang, Q.; Sun, Y.; Gu, S.; He, Z.; Wang, Q.; Zhang, F. Effect
of quenching temperature on sulfide stress cracking behavior of
martensitic steel. \emph{Materials Science and Engineering: A},
\textbf{2018}, 724, pp. 131-141.

{[}7{]}. ``Magnetic Properties of Ferritic, Martensitic and Duplex
Stainless Steels.'' \emph{British Stainless Steel Association}, British
Stainless Steel Association, 2018, www.bssa.org.uk/topics.php?article=3.

\textbf{Appendix}

\emph{A. Small sample hardness measurements}

\begin{longtable}[]{@{}llllll@{}}
\toprule
Composition & Sample \# & Hardness1 & Hardness2 & Hardness3 &
Hardness4\tabularnewline
\midrule
\endhead
A (1018) & 1 & 21.5 & 23.5 & 19.5 & 20\tabularnewline
A (1018) & 2 & 11.5 & 14 & 18 & 14\tabularnewline
A (1018) & 3 & 28.0 & 26 & 32 & 31\tabularnewline
A (1018) & 4 & 87.0 & 95 & 94 & 94\tabularnewline
A (1018) & 5 & 86.0 & 91 & 85 & 91\tabularnewline
A (1018) & 6 & 86 & 88 & 88.0 & 91.0\tabularnewline
A (1018) & 7 & 89.0 & 93 & 93.5 & 92.5\tabularnewline
A (1018) & 8 & 76.0 & 77.5 & 78 & 82\tabularnewline
C (4140) & 9 & 53.0 & 56 & 57 & 59\tabularnewline
C (4140) & 10 & 44.0 & 47 & 47.5 &\tabularnewline
C (4140) & 11 & 43.0 & 44 & 43 & 42\tabularnewline
C (4140) & 12 & 21.0 & 23 & 19 & 20\tabularnewline
C (4140) & 13 & 9.0 & 5 & 2 & 8\tabularnewline
C (4140) & 14 & 26.0 & 22 & 24 & 26\tabularnewline
C (4140) & 15 & 25.0 & 27 & 29 & 26\tabularnewline
C (4140) & 16 & 89.0 & 91 & 91 & 93\tabularnewline
B (1045) & 17 & 57.5 & 57 & 54 & 59\tabularnewline
B (1045) & 18 & 47.0 & 49 & 46 & 49\tabularnewline
B (1045) & 19 & 40.0 & 38.5 & 40.5 & 43\tabularnewline
B (1045) & 20 & 12.5 & 12 & 11 & 13\tabularnewline
B (1045) & 21 & 3.5 & 4 & 3.5 & 4.0\tabularnewline
B (1045) & 22 & 37.5 & 39 & 41 & 37.5\tabularnewline
B (1045) & 23 & 43.0 & 51.5 & 50 & 56.5\tabularnewline
B (1045) & 24 & 0 & 0 & 0.0 & 0.0\tabularnewline
D (4340) & 25 & 41.0 & 42 & 45 & 46\tabularnewline
D (4340) & 26 & 43.0 & 45 & 47.5 & 44.5\tabularnewline
D (4340) & 27 & 41.0 & 42.5 & 40 & 40.5\tabularnewline
D (4340) & 28 & 19.0 & 21 & 22 & 18\tabularnewline
D (4340) & 29 & 8.5 & 9.5 & 7.5 & 9.5\tabularnewline
D (4340) & 30 & 13.5 & 16.5 & 12 & 12\tabularnewline
D (4340) & 31 & 29.0 & 29.5 & 31 & 25\tabularnewline
D (4340) & 32 & 11.0 & 13 & 14 & 15.5\tabularnewline
\bottomrule
\end{longtable}

\emph{B. Jominy hardenability test results}

\begin{longtable}[]{@{}lllll@{}}
\toprule
\textbf{Distance from end of bar (inches)} & \textbf{A (1018)} &
\textbf{B (1045)} & \textbf{C (4140)} & \textbf{D (4340)}\tabularnewline
\midrule
\endhead
0.125 & 85 & 2 & 11 & 23\tabularnewline
0.25 & 83 & 6 & 13 & 25\tabularnewline
0.375 & 79 & 10 & 15 & 28\tabularnewline
0.5 & 80 & 8 & 15.5 & 25\tabularnewline
0.625 & 94 & 7.5 & 19 & 26\tabularnewline
0.75 & 73 & 9 & 19 & 25\tabularnewline
0.875 & 97 & 9 & 20 & 25\tabularnewline
1 & 99 & 10 & 26 & 27\tabularnewline
1.125 & 72 & 10 & 21 & 23\tabularnewline
1.25 & 94 & 10 & 24 & 25\tabularnewline
1.5 & 90 & 9 & 19 & 27\tabularnewline
1.75 & 92 & 11 & 11 & 25\tabularnewline
2 & 95 & 11 & 20 & 27\tabularnewline
2.25 & 95 & 14.5 & 21 & 30.5\tabularnewline
2.5 & 98 & 14 & 14 & 30\tabularnewline
2.75 & 93 & 15 & 23 & 33\tabularnewline
3 & 95 & 16 & 25 & 32\tabularnewline
3.25 & 91 & & 22 &\tabularnewline
\bottomrule
\end{longtable}

\end{document}
